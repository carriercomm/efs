\section{日记}
\includegraphics[scale=0.1]{../images/mp/doing-6.mps}
\subsection{2010.1.1}
以后日记也采用Tex来写,开始会觉得受点约束,不灵活,待工具用熟了,自然来的畅快。

Ubuntu下没有好用的输入法,现在用的\ggemph{Python Pin Yin},常常死掉。

\LaTeX下对中文的完美支持是个难题,现在的方法显示出来不匀称,看起来颇粗糙,
先凑合着用,待对基本概念熟悉之后,有空寻找更好的中文支持方法。

\begin{verse}
verse example:
今天是2010年第一天,坐下来写这些,似乎预示了又一个繁忙的一年。
要妥善计划,时光的流逝,现在每每让人觉得太快了。
要做的事情太多,反而在忙忙碌碌中,找不到方向。精力时间都太有限,
不允许无谓的浪费。善用其心,也要善用其时。
\end{verse}

对2010年有什么期待呢?
\begin{quote}
第一当然是工作上要有起色,有目标,有成就出来。反思过去的几年,是太没成效了。
原因何在呢?首先要归因于没有明确的目标,二是学习方法和心态的不对头,
当然还有机遇的把握不够等等。“往者不可谏,来者犹可追。"
\end{quote}

在技术上的主要任务:
\begin{itemize}
  \item 建模技术\footnote{图论,PN,FSM,UML 图形化建模技术及其分析方法}。
  \item 算法的研究\footnote{sort and search, kinds of data structrues,并行算法和分布式算法}
  \item 系统软件\marginpar{操作系统,文件系统,数据库系统,编译原理和计算机网络}。
  \item 对业界的理解。多接触不同的人和事,看看别人都在做什么,怎么做的,从中学习。
\end{itemize}

另外,把英语的学习抓起来,主要是听说方面,即正常的交流应用方面。读写方面兼顾。

\begin{verbatim}
#include <stdio.h>

int main()
{
        printf(``Hello,world\n'');
        return 0;
}
\end{verbatim}

\subsection{2010.1.2}
