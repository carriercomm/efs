\newpage
\section{Introduction}
你好,\TeX, \LaTeX, \LaTeXe

这是一个很好的工具,要通过实践好好掌握。 learning by doing is the best
method,通过实践来学习是最好的学习方法。\underline{先掌握基本的形式},逐步的推衍成复杂的系统。

http://www.emule.org.cn/server.met

http://www.edonkey2000.com/server.met

\emph{An effective executive} does not need to be a leader in the sense that
the term is now most commonly used. Harry Truman did not have one ounce of
charisma, for example, yet he was among the most effective chief executives in
U.S. history. Similarly, some of the best business and nonprifit CEOs I've
worked with over a 65-year consulting career were not stereotypical leaders.
They were all over the map in terms of their personalities, attitudes, values,
strengths, and weeknesses.  They ranged from extroverted to nearly reclusive,
from easygoing to controlling, from generous to parsimonious.

What made them all effective is that they followed the same eight practices:

\begin{itemize}
  \item They asked, ``What needs to be done?''
  \item They asked, ``What is right for the enterprise?''
  \item They developed action plans.
  \item They took responsibility for decisions.
  \item They took responsibility for communicating.
  \item They were focused on opportunities rather than problems.
  \item They ran productive meetings.
  \item They thought and said ``we'' rather than ``I''.
\end{itemize}

\begin{enumerate}
  \item They asked, ``What needs to be done?''
  \item They asked, ``What is right for the enterprise?''
  \item They developed action plans.
  \item They took responsibility for decisions.
  \item They took responsibility for communicating.
  \item They were focused on opportunities rather than problems.
  \item They ran productive meetings.
  \item They thought and said ``we'' rather than ``I''.
\end{enumerate}

The first two practices gave them the knowledge they needed. The next
four helped them convert this knowledge into effective action. The last
two ensured that the whole organization felt responsible and accoutable.

\newpage
\section{Tables}

The first Table is\\
\begin{tabular}{r|l|c} 
  \hline
  position & club             & games\\[0.5ex] \hline
  1        & amsville rockets & 33\\
  2        & borden comets    & 33\\
  3        & quincy giants    & 55\\
  \hline
\end{tabular}

The second table is\\
\begin{tabular}
  {|r|l||c|rrr|c|c|} \hline
  Position & Club              & Games & W  & T  & L & Goals & Points\\ \hline\hline
  1        & Amesville Rockets & 33    & 19 & 13 & 1 & 66:31 & 51:15\\ \hline
  2        & Amesville Rockets & 33    & 19 & 13 & 1 & 66:31 & 51:15\\ \hline
  3        & Amesville Rockets & 33    & 19 & 13 & 1 & 66:31 & 51:15\\ \hline
\end{tabular}

\newpage
\section{Boxes}
\begin{theorem}
  Every infinite set of bounded points possesses at least one maximum point.
\end{theorem}

\begin{alltt}
Underlining \underline{typewriter} text is also possible.
Note that dollar($) and percent (%) signs are treated \emph{literally}.
\end{alltt}

\begin{minipage}[t][\totalheight][t]{3cm}
  This is a minipage of height of 2~cm with the text at the top.
  \includegraphics[scale=0.1]{../images/mp/doing-6.mps}
\end{minipage}\hrulefill
\parbox[t][\totalheight][c]{3cm}{In this parbox, the text is centered on the same height.}\hrulefill
\begin{minipage}[t][\totalheight][b]{3cm}
  In this third paragraph box, the text is at the bottom\ggemph{minipage}.
\end{minipage}

\noindent
baseline\rule{1cm}{0.01cm}\\
baseline\rule{1cm}{0cm}\\*[5cm]
baseline\rule{1cm}{0.1cm}\\
baseline\rule{1cm}{1cm}\\

\noindent 
package fancybox.\marginpar{package fancybox}, there are some examples:\\
\shadowbox{\parbox{12cm}{The width of the shadom is given by the length $\backslash$shadowsize, default 4pt.
Multiline text must be placed in a minipage environment, the same as for $\backslash$fbox.}}

\noindent 
\doublebox{\parbox{12cm}{The width of the inner frame is 0.75$\backslash$fboxrule, that of the outer frame is 
1.5$\backslash$fboxrule, and the spacing between the frames is 1.5$\backslash$fboxrule plus 0.5pt.}}

\begin{flushright}
2010年的第二天,去理发。
\end{flushright}

\subsection{Box}
这里是Box的一些实例:
\makebox[5cm]{some words} \par
\framebox[5cm][r]{some words}

首行若有缩进,是由段落缩进造成的\\
\mbox{mbox test, LR盒子的一种,其它还有fbox, makebox, framebox}\\
\fbox{fbox test, LR盒子的一种,其它还有fbox, makebox, framebox}\\
\makebox{makebox test, LR盒子的一种,其它还有fbox, makebox, framebox}\\
\framebox{framebox test, LR盒子的一种,其它还有fbox, makebox, framebox}\\

首行若有缩进,是由段落缩进造成的\\
\framebox[2\width][r]{两倍宽度}\\
\fbox{没有支撑}\\ 
\fbox{使用支撑\rule[-3mm]{0pt}{8mm}}\\ 

首行若有缩进,是由段落缩进造成的\\
\parbox{12cm}{在\TeX中一切都是盒子,单个字符是盒子,若干字符排成一行构成一个大一点的行盒子,
若干行盒子堆叠成段落盒子和页盒子。}

首行若有缩进,是由段落缩进造成的\\
Left\hrulefill Right\\
Left\hrulefill\\
Left\dotfill Right\\
Left\dotfill\\
\begin{minipage}{12cm}{在\TeX中一切都是盒子,单个字符是盒子,若干字符排成一行构成一个大一点的行盒子,
若干行盒子堆叠成段落盒子和页盒子。}
\end{minipage}

\newpage
\section{Tools}
下决心好好学习如下几类工具,写文档的用\LaTeX/MetaPost,代码编辑用Vim,Make,Git,GDB,学习的方法是慢慢积累,从简单到复杂,
主要还是对核心概念的深入体会,贯通起来,在实际应用中不断完善。

工具链

先从MetaPost下手, 待用的熟练了,在细致的学习\LaTeX。难点在中文的支持,需要理解很多技术细节。 工具用熟了,就可以转入具体的文章謀篇布局的讲究了。

\newpage
\section{MetaPost}
latex+dvipdfmx接受的图形格式为ps,eps, pdflatex接受的图形格式为png.首先通过MetaPost生成图形,然后插进来。

生成图形的方式有多种

includegraphics

\LaTeX{} is a document preparation system for the \TeX{} typesetting program. 
Few people write in plain \TeX{} anymore. The current version is \LaTeXe.

\begin{align}
  E &= mc^2\\
  m &= \frac{m_0}{\sqrt{1-\frac{v^2}{c^2}}}
\end{align}

