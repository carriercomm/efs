\section{YFS}

\subsection{Process and Communication}
C/S Model, Task(Request/Reply)的stage,即pipeline. 

network protocols

各进程间的链接管理,主要建立在TCP链接管理之上,链接建立和撤销,对各种网络事件的处理
链接状态的维护和变迁图,即状态机模型。

数据流,或者说data path, data flow, or communication path. 

hello请求区别于特定的请求,在于它是由网络层自己处理的,不需要应用层的callback
目前有两类hello request: selfcheck and getinfo.

为什么监听一个固定端口,只接受hello,然后连接到一个随机监听的端口。
整个cluster中,只有一个节点如此处理,分配NID,别的节点的INFO信息,都由该节点存储,并提供查询接口.

每一个server的client端,都提供两套接口:blocked and unblocked. 对blocked接口,在网络层收到响应后,会以信令的方式通知。
\begin{verbatim}
sem_timedwait
sem_post
\end{verbatim}

ynet-based server architecture. for example: cds, mds, c60d

\subsection{Naming}

为了实现持久通信,消息必须具有某种不变性,消息的组成:消息头和消息体。消息头的意义
比如目标地址标示NID具有全局唯一性和不变性

\begin{verbatim}
rpc_host2nid
\end{verbatim}

\subsection{Resource Management}
Resource View.

concurrency and synchronization

replication and consistency



