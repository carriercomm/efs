% study.tex

\section{道德经}
\subsection{道经}

\ggnewpage {
道\footnote{道有体用,其体杳杳冥冥,其用神妙不测}
可道,非常道;名可名,非常名。
无,名天地之始;有,名万物之母。
故常无,以观其妙;常有,以观其缴。
此两者同出而异名,同谓之玄,
玄之又玄,众妙之门。
}

\ggnewpage {
天下皆知美之为美,斯恶已;皆知善之为善,斯不善矣。
故\textcolor{blue}{有无相生,难易相成,长短相形,高下相倾,音声相和,前后相随}。
}

\ggnewpage {
天长地久。天地\footnote{人的意志投射到自然界}
之所以能长且久者,以其不自生,故能长生。
是以圣人后其身而身先,外其身而身存,非以其无私耶?故能成其私。
}

\ggnewpage {
上善若水\footnote{总摄七善}。水善利万物而不争。
处众人之所恶,故几于道。
\textcolor{blue}{居善地,心善渊,与善仁,言善信,政善治,事善能,动善时}\footnote{七善}。
夫唯不争,故天下莫能与之争。
}

\subsection{德经}

\newpage
\section{The Art Of Learning}
\ggnewpage {
有效地压缩技能的外在表现同时又紧紧围绕技能的内在实质。
一段时间之后,广度就会慢慢缩小而力量则会逐渐增加。
}

\ggnewpage {
首先从基础开始,通过\em{理解训练的原则所在来建立扎实的基础},然后在你个人倾向的指导下
拓宽并完善自己的技能,同时和你认为是艺术的必要实质的东西在抽象层面上保持联系。
使得这些从个人的着重点中拓展的知识相互连接成一个网络。\footnote{pp105-106}
}

\section{BLOG}
从今天开始用Latex写各种文档,逐渐加深对它的掌握。绘制图形的方法是个难点和重点,但既然
未必用得到,可以从简单的学起。其它方面的学习也是如此,\textcolor{red}{把困难的问题分解成可以轻松掌握
的部分,各个击破,然后有望在一个高的层面上加以融合贯通}。

在事上磨练,就是理性认识\colorbox{red}{从抽象到具体的过程},不要在抽象的认识阶段迷失了方向,悬空揣摩文义,
老死句下,而不得立言宗旨所在,永远的处于惑的状态。

儒释道的学问有相通处,举例如下:
\begin{itemize}
  \item 洪范九筹
  \item 大学三纲八目
  \item 中庸问政九经
  \item 老子七善之道
  \item 文子九守
  \item 华严十地品
\end{itemize}

继续YFS方面的开发,也是不错的选择,顺便把基本的知识,再加以深化。基本点有:
\begin{itemize}
  \item C/Python
  \item Linux
  \item Network
  \item Distributed Algorithm
\end{itemize}
如何开发一个高性能的Server是面临的核心问题(\fbox{C10K})。在设计方面有许多的情况要考虑,会面对各种
设计上的权衡。
